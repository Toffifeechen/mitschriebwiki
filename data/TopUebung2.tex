\documentclass{article}
\usepackage[utf8]{inputenc}
\usepackage{amsmath}
\usepackage{amsfonts}
\usepackage{amssymb}
\usepackage{amsthm}
\usepackage{mathrsfs}
\usepackage{german}
\usepackage{enumerate}
\usepackage{stmaryrd}
\title{2. Topologie Übung}
\author{Ferdinand Szekeresch}
\date{31. Oktober 2007}
\begin{document}
\maketitle

\textbf{Nachtrag}\\
$F$ heißt Faserprodukt von $A$ und $B$ über $S$, $:\Leftrightarrow$ wenn für jede Menge $M$ und jedes Paar von Abb. $g_A, g_B$ nach $A$ bzw. $B$ mit $f_A\circ g_A = f_B\circ g_B$ genau eine Abb. $h: M\rightarrow F$ ex. mit $g_A = \pi_a\circ h, g_B = \pi_B \circ h$.\\

\textbf{Aufgabe 1}\\
$(X,d),(Y,d)$ metr. Räume, $f_2: X\times Y \rightarrow \mathbb{R}_\geq$ gegeben durch ÜB\\
Behauptung: $F_2$ ist Metrik.\\
\begin{itemize}
\item Symmetrie: klar
\item Definitheit: klar, da $d,e$ Metriken.
\item Dreiecksungleichung: Sei $(x_1,y_1),(x_2,y_2),(x_3,y_3) \in X\times Y$\\
$\Rightarrow f_2\big((x_1,y_1),(x_3,y_3)\big) = \big(d(x_1,x_3)^2 + e(y_1),y_3)^2\big)^\frac12\\
\leq \Big(\big(d(x_1,x_2)+d(x_2,x_3)\big)^2 + \big(e(y_1,y_2)+e(y_2,y_3)\big)^2\Big)^\frac12\\
\leq\big(d(x_1,x_2)^2 + e(y_1,y_2)^2\big)^\frac12 + \big(d(x_2,x_3)^2 + e(y_2,y_3)^2\big)^\frac12$\\
%Klammern und f_2 drunterschreiben
\end{itemize}
$\Rightarrow f_2$ ist Metrik.\\
2te Metrik
\begin{itemize}
\item Symmetrie: klar
\item Definitheit: klar
\item Dreiecksungl. Sei $(x_1,y_1),(x_2,y_2),(x_3,y_3) \in X\times Y$\\
$\Rightarrow f_\infty\big((x_1,y_1),(x_3,y_3)\big) = \max\big(d(x_1,x_3),e(y_1,y_3)\big)\\
\leq \max\big(d(x_1,x_2)+d(x_2,x_3), e(y_1,y_2)+e(y_2,y_3)\big)\\
\leq \max\big(d(x_1,x_2),e(y_1,y_2)\big)+ \max\big(d(x_2,x_3),e(x_2,x_3)\big)$\\
%Klammern und f_\infty drunter
\end{itemize}
$\Rightarrow f_\infty$ ist Metrik.\\
Gesucht: $c>0$ mit: $\forall(x_1,y_1),(x_2,y_2)\in X\times Y: \frac1cf_2(\ldots)\leq f\infty(\ldots)\leq cf_2(\ldots)$.\\
z.B. $\sqrt 2$ erfüllt das, denn: Seien $x,y \in\mathbb{R}_{\geq0}$, dann gilt\begin{itemize}
\item $(x^2+y^2)^\frac12 \leq \big(\max(x,y)^2+\max(x,y)^2\big)^\frac12 = \big(2\max(x,y)^2\big)^\frac12 = \sqrt 2\max(x,y)\\
\Rightarrow (1)$
\item $\max(x,y) = \sqrt{\max(x,y)^2} \leq \sqrt{x^2+y^2}\leq\sqrt2\sqrt{x^2+y^2}\Rightarrow (2)$
\end{itemize}

\textbf{Aufgabe 2}\\
Definiere $D:=X\uplus Y :=(X\times\{0\})\cup(Y\times\{1\})$\\
Disjunkte Vereinigung von $X$ und $Y$.\\
$f: X\rightarrow D, x\mapsto(x,0)\\
g: Y\rightarrow D,y\mapsto(y,1)$.\\
Definiere $c: D\rightarrow Z$ durch $m\mapsto\left\{\begin{array}{ll}k(x) & ,\text{falls} m=(x,0)\\ l(y) & ,\text{falls} m=(y,1)\end{array}\right.$\\
$c$ erfüllt das Gewünschte. Soll gelten:\\
$\forall x\in X: c(f(x)) = k(x)\\
\forall y\in Y: c(g(y)) = l(y)$, dann muss nach Def. von $f,g$ gelten:\\
$c((x,0)) = k(x)\\
c((y,1)) = l(y)$.\\
Seien $D_1,D_2$ Mengen mit obiger Eigenschaft\\
$\Rightarrow \begin{array}{llll}\exists ! c_1:D_1\rightarrow D_2 : & f_2 = c_1\circ f_1\\ & g_2 = c_1\circ g_1\\
\exists ! c_2 : D_2\rightarrow D_1 : & f_1 = c_2\circ f_2 \\ & g_2 = c_1\circ g_1\end{array}$\\
$\Rightarrow c_2\circ c_1$ ist Abb. mit $c_2\circ c_1\circ f_1 = f_1, c_2 \circ c_1\circ g_1 = g_1$\\
$\Rightarrow c_2\circ c_1 = \text{id}_{D_1}$ genauso $c_1\circ c_2 = \text{id}_{D_2}$\\
$c_1$ ist Bijektion zw. $D_1$ und $D_2$, und zwar die einzig sinnvolle.\\

\textbf{Aufgabe 3}\\
Beweis:\\
Nachrechnen: $\left(\begin{array}{cc}1 & 0 \\ 0 & 1\end{array}\right)\bullet z = z\\
\left(\left(\begin{array}{cc}a & b\\c & d\end{array}\right)\left(\begin{array}{cc}\tilde a & \tilde b \\ \tilde c & \tilde d\end{array}\right)\right)\bullet z = \left(\begin{array}{cc}a & b\\c & d\end{array}\right)\left(\left(\begin{array}{cc}\tilde a & \tilde b\\\tilde c &\tilde d\end{array}\right)\bullet z\right)$\\
Noch zu zeigen:\\
$\left(\begin{array}{cc}a & b\\c & d\end{array}\right)\bullet z \in L\backslash K$ für alle $z\in L\backslash K$.\\
Wäre $\left(\begin{array}{cc}a & b\\c & d\end{array}\right)\bullet z = \frac{az +b}{cz + d} = k\in K$ dann wäre das äquivalent zu\\
$(a-ck)z = dk-b \Leftrightarrow a-ck = 0 = dk-b \Rightarrow \left(\begin{array}{c}a\\b\end{array}\right) = k\left(\begin{array}{c}c\\d\end{array}\right) \Rightarrow \left(\begin{array}{c}a\\b\end{array}\right),\left(\begin{array}{c}c\\d\end{array}\right)$ lin. abh. über $K \rightarrow \left(\begin{array}{cc}a & b\\c & d\end{array}\right)\notin \text{GL}_2(K)$\\
Sei $K = \mathbb{R}, L=\mathbb{C}$, dann ist die Operation transitiv, denn:\\
Alle $z\in\mathbb{C}\backslash\mathbb{R}$ liegen in der Bahn von $i$.\\
Sei $z=x+iy \in\mathbb{C}\backslash\mathbb{R}$, dann ist $\left(\begin{array}{cc}y & x\\0 & 1\end{array}\right) = \frac{yi+x}{1} = x+iy = z$.\\
Welche Elemente aus $\text{GL}_2(\mathbb{R})$ haben Fixpunkte in $\mathbb{C}\backslash\mathbb{R}$?\\
Sei $A=\left(\begin{array}{cc}a & b\\c & d\end{array}\right)$\begin{itemize}
\item[1.Fall] $c\neq 0$. Dann gilt $\left(\begin{array}{cc}a & b\\c & d\end{array}\right)\bullet z = z \Leftrightarrow \frac{az+b}{cz+d} = z\\
\Leftrightarrow cz^2 + (d-a)z -b = 0$\\
Mitternachtsformel: Diese Gleichung hat\begin{enumerate}[(i)]
\item 1 reelle Lösung, falls $(a-d)^2 + 4bc = 0$
\item 2, reelle Lösungen, falls $(a-d)^2 + 4bc > 0$
\item 2 echt komplexe Lösungen, falls $(a-d)^2 + 4bc < 0$
\end{enumerate}
(iii) $\Leftrightarrow \text{Spur}(A)^2 < 4\det(A)$
\item[2.Fall] $c=0$\begin{enumerate}[(i)]
\item $a=d : \left(\begin{array}{cc}a & b\\0 & a\end{array}\right)\bullet z = z \Leftrightarrow az+b=az\Leftrightarrow b=0$\\
Daraus folgt $\left(\begin{array}{cc}a & 0\\0 & a\end{array}\right)$ fixiert alle Elemente aus $\mathbb{C}\backslash\mathbb{R}$
\item $a\neq d: \left(\begin{array}{cc}a & b \\0 & d\end{array}\right)\bullet z = z \Leftrightarrow \frac{az+b}d = \frac{b}{d-a}\in\mathbb{R}\Rightarrow A$ hat keine Fixpunkte in $\mathbb{C}\backslash\mathbb{R}$\\
\end{enumerate}
\end{itemize}
Insgesamt: Fixpunkte in $\mathbb{C}\backslash\mathbb{R}$ haben Matrizen der Form $\left(\begin{array}{cc}r & 0\\0 & r\end{array}\right)$ und Matrizen mit $\text{Spur}A^2<4\det A$
\end{document}
